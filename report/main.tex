\documentclass[letterpaper, twocolumn]{article}
\usepackage{url,graphicx,xcolor}
\usepackage{times}
\usepackage{helvet}
\usepackage{amsmath}
\usepackage{courier}
\usepackage[]{faikrmod3}
\usepackage{float}
\usepackage{booktabs}
\usepackage{adjustbox}

\frenchspacing
\setlength{\pdfpagewidth}{8.5in}
\setlength{\pdfpageheight}{11in}

\pdfinfo{
/Title (Bayesian Network Modeling for Holistic Injury Risk Prediction in Athletes)
/Author (Davide Boschi, Gabriele Napoletano, Davide Giannetti)
}
\setcounter{secnumdepth}{0}

\begin{document}

\title{Bayesian Network Modeling for Holistic Injury Risk Prediction in Athletes}
\author{
Davide Boschi, Gabriele Napoletano, Davide Giannetti\\
Master's Degree in Artificial Intelligence, University of Bologna\\
\{davide.boschi3, gabriele.napoletano, davide.giannetti\}@studio.unibo.it
}
\maketitle

\begin{abstract}
\begin{quote}
This project develops a Bayesian Network to analyse daily injury risk in endurance athletes. The model integrates static characteristics, wellness indicators, training load, and injury history. Using a synthetic longitudinal dataset of 366,000 samples, we defined a domain-informed DAG and learned parameters via Bayesian Estimation with K2 smoothing. To address the class imbalance inherent in injury prediction ($\approx$8.8\% positive cases), we implemented a risk-averse validation strategy, optimizing the classification threshold (20\%) to prioritize sensitivity. The model achieves a Recall of 66.5\% and a Precision of 29.3\% on unseen test data, successfully intercepting the majority of injury risks. Diagnostic inference identifies recent injury history (89.3\%) as the dominant predictor, validating the "load paradox" hypothesis.
\end{quote}
\end{abstract}

\section{Introduction}

\subsection{Domain}
Endurance athletes face injury risks driven by complex, non-linear interactions between training load, recovery, and physiological traits. Traditional monitoring often isolates these metrics, failing to capture compound effects \cite{rossi2025}. For example, a high training load may be sustainable for a rested athlete but injurious for one experiencing high life stress. We model this domain using a probabilistic approach to capture conditional dependencies between static traits (e.g., BMI) and dynamic stressors.

\subsection{Aim}
The aim of this study is to build a Bayesian Network (BN) to formalise these dependencies and quantify daily injury probability ($P(\text{Injury}=\text{Yes}|\mathbf{Evidence})$). Specifically, we aim to:
\begin{itemize}
    \item Investigate how static factors (like BMI) amplify dynamic risks (like Stress).
    \item Calibrate the model to maximize injury detection (Recall) rather than standard accuracy, addressing the cost-sensitive nature of medical prediction.
    \item Perform diagnostic inference to identify the root causes of injuries (e.g., Load vs. History).
\end{itemize}

\subsection{Method}
We designed a Directed Acyclic Graph (DAG) based on sports science literature. The hierarchy flows from Static Factors to Wellness indicators, which modulate Load Metrics, finally determining Injury Risk.

We used the \texttt{pgmpy} library for implementation. To handle the longitudinal nature of the data, we performed Time-Series Feature Engineering (e.g., rolling averages for ACWR), transforming daily logs into probabilistic risk snapshots.
The synthetic dataset was split into Training (80\%) and Test (20\%) sets to ensure robust validation. To prevent zero-probability issues in rare scenarios, Conditional Probability Tables (CPTs) were learned using Bayesian Estimation with K2 Smoothing. Inference was performed using Variable Elimination.

\section{Model}

\begin{figure}[h]
    \centering
    \includegraphics[width=0.95\columnwidth]{network_image.png}
    \caption{Bayesian Network topology. Static factors influence wellness, which modulates the impact of training load.}
    \label{fig:network}
\end{figure}

The network (Figure \ref{fig:network}) consists of 15 discrete variables organized into four functional groups:
\begin{itemize}
    \item \textbf{Static (Long-term):} Age\_d, Sex\_d, BMI\_d, Training\_Experience, Lifestyle\_Factor. These define baseline susceptibility.
    \item \textbf{Wellness (Daily):} Sleep\_Quality, Morning\_HRV, Daily\_Stress. These capture the daily physiological state.
    \item \textbf{Load (Training):} Avg\_Intensity, Training\_Load. Direct measures of work performed.
    \item \textbf{Hubs \& History:} Ready\_To\_Train, Acute\_Chronic\_Ratio (ACWR), Injury\_Prev30d, Injury\_History\_12m. Computed nodes representing readiness and vulnerability.
    \item \textbf{Target:} Injury\_Risk. The binary outcome.
\end{itemize}

\section{Analysis}

\subsection{Experimental setup}
We conducted two types of inference on the Test set:
\begin{enumerate}
    \item \textbf{Performance Validation:} We compared standard prediction (threshold 0.5) against a Risk-Averse strategy (threshold 0.2) to optimize the Recall/Precision trade-off.
    \item \textbf{Predictive Scenarios:} Simulating athlete profiles to quantify risk probabilities.
    \item \textbf{Diagnostic Inference:} Setting $Injury\_Risk = Yes$ to query the posterior probabilities of parent nodes (Root Cause Analysis).
\end{enumerate}

\subsection{Results}

\textbf{Performance \& Validation.}
The model demonstrates high robustness with a finite Log-Likelihood on the test set (-911,635.22).
As expected with imbalanced data, the standard threshold yielded high accuracy (91.7\%) but very low sensitivity. By lowering the threshold to 20\%, we achieved a Recall of 66.5\% (Precision 29.3\%, Accuracy 83.9\%). This means the model correctly identifies 2 out of 3 injuries. The trade-off in precision is clinically justified, as false alarms are less costly than missed injuries in preventive medicine.

\textbf{Predictive Analysis.} Table \ref{tab:scenarios} shows risk probability across scenarios. The baseline risk is 11.17\%.

\begin{table}[h]
\centering
\small
\begin{adjustbox}{max width=\columnwidth}
\begin{tabular}{l|l|c}
\toprule
\textbf{Scenario} & \textbf{Configuration} & \textbf{Risk P(Yes)} \\
\midrule
Bulletproof & Normal BMI, Great Sleep, Low Load, No Hist. & 1.22\% \\
Positive Life & Positive Lifestyle & 9.55\% \\
Baseline & No Evidence (Average Athlete) & 11.17\% \\
Negative Life & Negative Lifestyle & 12.72\% \\
Fragile & Underweight, High Stress & 14.33\% \\
Overloaded & Obese, High Stress & 30.00\% \\
Perfect Storm & Obese, Poor Sleep, High Load, Recent Inj. & 40.00\% \\
\bottomrule
\end{tabular}
\end{adjustbox}
\caption{Predictive Risk Stratification. The model discriminates between safe and critical states by a factor of $\approx 33x$. Lifestyle alone accounts for a $\sim25\%$ relative risk shift.}
\label{tab:scenarios}
\end{table}

A key insight is the Mechanical vs. Metabolic comparison. While being underweight increases risk moderately (14.3\%), being obese under the same high-stress conditions doubles that risk (30.0\%), suggesting a non-linear interaction between body mass and stress.

\textbf{Diagnostic Analysis.} By querying the causes of an injury event (Table \ref{tab:diagnostic}), we established a hierarchy of risk factors.

\begin{table}[h]
\centering
\small
\begin{adjustbox}{max width=\columnwidth}
\begin{tabular}{l|c|l}
\toprule
\textbf{Factor} & \textbf{P(Factor|Injury)} & \textbf{Interpretation} \\
\midrule
Recent History & 89.3\% & Primary Cause (Recurrence) \\
Low HRV & 43.1\% & Physiological Stress Indicator \\
High Load & 21.1\% & Trigger (Spike) \\
Safe Load & 67.6\% & The Load Paradox \\
\bottomrule
\end{tabular}
\end{adjustbox}
\caption{Posterior probabilities of risk factors given an injury.}
\label{tab:diagnostic}
\end{table}

The results validate the "Load Paradox". Only 21.1\% of injuries were associated with "High" acute workload spikes. The vast majority occurred under "Safe/Low" loads, but were strongly linked to compromised capacity (Recent Injury History 89.3\% and Low HRV), implying that most injuries are recurrences triggered by normal loads on a fragile system.

\section{Conclusion}
This work demonstrates that a Bayesian Network can effectively model athletic injury risk using synthetic data.
1. \textbf{Valid:} The model generalizes well (Test Set Recall 66.5\%).
2. \textbf{Discriminatory:} It distinguishes between "Bulletproof" (1.2\%) and "Critical" (40.0\%) states.
3. \textbf{Actionable:} It identifies that prevention should prioritise managing return-to-play (History) and lifestyle factors (Sleep/Stress) over simple load restriction.

\section{Links to external resources}
\begin{itemize}
    \item Dataset: \url{https://zenodo.org/records/15401061}
    \item Reference Paper: Rossi \& Rodrigues (2025)
\end{itemize}

\bigskip
\bibliographystyle{aaai}
\bibliography{faikrmod3.bib}

\end{document}