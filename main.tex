\documentclass[letterpaper, twocolumn]{article}
\usepackage{url,graphicx,xcolor}
\usepackage{times}
\usepackage{helvet}
\usepackage{amsmath}
\usepackage{courier}
\usepackage[]{faikrmod3}
\usepackage{float}
\usepackage{booktabs}
\usepackage{adjustbox}

\frenchspacing
\setlength{\pdfpagewidth}{8.5in}
\setlength{\pdfpageheight}{11in}

\pdfinfo{
/Title (Bayesian Network Modeling for Holistic Injury Risk Prediction in Athletes)
/Author (Davide Boschi, Gabriele Napoletano, Davide Giannetti)
}
\setcounter{secnumdepth}{0}

\begin{document}

\title{Bayesian Network Modeling for Holistic Injury Risk Prediction in Athletes}
\author{
Davide Boschi, Gabriele Napoletano, Davide Giannetti\\
Master's Degree in Artificial Intelligence, University of Bologna\\
\{davide.boschi3, gabriele.napoletano, davide.giannetti\}@studio.unibo.it
}
\maketitle

\begin{abstract}
\begin{quote}
This project develops a Bayesian Network to analyse daily injury risk in endurance athletes. The model integrates static characteristics, wellness indicators, training load, and injury history. Using a longitudinal dataset of 366,000 samples, we defined a domain-informed DAG and learned parameters via maximum likelihood. To address the class imbalance inherent in injury prediction ($\approx$8\% positive cases), we implemented a risk-averse validation strategy, optimizing the classification threshold (20\%) to prioritize sensitivity over raw accuracy. The model achieves a Recall of 69\% on unseen test data, successfully intercepting the majority of injury risks. Diagnostic inference identifies recent injury history (89.5\%) as the dominant predictor, validating the "load paradox" hypothesis.
\end{quote}
\end{abstract}

\section{Introduction}

\subsection{Domain}
Endurance athletes face injury risks driven by complex, non-linear interactions between training load, recovery, and physiological traits. Traditional monitoring often isolates these metrics, failing to capture compound effects \cite{rossi2025}. For example, a high training load may be sustainable for a rested athlete but injurious for one experiencing high life stress. We model this domain using a probabilistic approach to capture conditional dependencies between static traits (e.g., BMI) and dynamic stressors.

\subsection{Aim}
The aim of this study is to build a Bayesian Network (BN) to formalise these dependencies and quantify daily injury probability ($P(\text{Injury}=\text{Yes}|\mathbf{Evidence})$). Specifically, we aim to:
\begin{itemize}
    \item Investigate how static factors (like BMI) amplify dynamic risks (like Stress).
    \item Calibrate the model to maximize injury detection (Recall) rather than standard accuracy, addressing the cost-sensitive nature of medical prediction.
    \item Perform diagnostic inference to identify the root causes of injuries (e.g., Load vs. History).
\end{itemize}

\subsection{Method}
We designed a Directed Acyclic Graph (DAG) based on sports science literature. The hierarchy flows from Static Factors to Wellness indicators, which modulate Load Metrics, finally determining Injury Risk.

We used the \texttt{pgmpy} library for implementation. To handle the longitudinal nature of the data, we performed \textbf{Time-Series Feature Engineering} (e.g., rolling averages for ACWR), transforming daily logs into probabilistic risk snapshots.
The dataset was split into Training (80\%) and Test (20\%) sets to ensure robust validation. Conditional Probability Tables (CPTs) were learned using Maximum Likelihood Estimation (MLE). Inference was performed using Variable Elimination.

\section{Model}

\begin{figure}[h]
    \centering
    \includegraphics[width=0.95\columnwidth]{network_image.png}
    \caption{Bayesian Network topology. Static factors influence wellness, which modulates the impact of training load.}
    \label{fig:network}
\end{figure}

The network (Figure \ref{fig:network}) consists of 15 discrete variables organized into four functional groups:
\begin{itemize}
    \item \textbf{Static (Long-term):} \textit{Age\_d, Sex\_d, BMI\_d, Training\_Experience, Lifestyle\_Factor}. These define baseline susceptibility.
    \item \textbf{Wellness (Daily):} \textit{Sleep\_Quality, Morning\_HRV, Daily\_Stress}. These capture the daily physiological state.
    \item \textbf{Load (Training):} \textit{Avg\_Intensity, Training\_Load}. Direct measures of work performed.
    \item \textbf{Hubs \& History:} \textit{Ready\_To\_Train, Acute\_Chronic\_Ratio (ACWR), Injury\_Prev30d, Injury\_History\_12m}. Computed nodes representing readiness and vulnerability.
    \item \textbf{Target:} \textit{Injury\_Risk}. The binary outcome.
\end{itemize}

\section{Analysis}

\subsection{Experimental setup}
We conducted two types of inference on the Test set:
\begin{enumerate}
    \item \textbf{Performance Validation:} We compared standard prediction (threshold 0.5) against a **Risk-Averse strategy** (threshold 0.2) to optimize the Recall/Precision trade-off.
    \item \textbf{Predictive Scenarios:} Simulating athlete profiles to quantify risk probabilities.
    \item \textbf{Diagnostic Inference:} Setting $Injury\_Risk = Yes$ to query the posterior probabilities of parent nodes (Root Cause Analysis).
\end{enumerate}

\subsection{Results}

\textbf{Performance \& Validation.}
The model demonstrates high robustness, with negligible difference between Train and Test accuracy ($<0.2\%$), ruling out overfitting.
As expected with imbalanced data, the standard threshold yielded high accuracy (91.7\%) but zero sensitivity. By lowering the threshold to 20\%, we achieved a \textbf{Recall of 69\%} (Accuracy 83.2\%), effectively predicting 4,187 out of 6,068 injuries in the test set. This trade-off is justified as false alarms are less costly than missed injuries in preventive medicine.

\textbf{Predictive Analysis.} Table \ref{tab:scenarios} shows risk probability across scenarios. The baseline risk is 8.73\%.

\begin{table}[h]
\centering
\small
\begin{adjustbox}{max width=\columnwidth}
\begin{tabular}{l|l|c}
\toprule
\textbf{Scenario} & \textbf{Configuration} & \textbf{Risk P(Yes)} \\
\midrule
\textit{Bulletproof} & Normal BMI, Great Sleep, Low Load, No Hist. & \textbf{1.14\%} \\
\textit{Ideal} & Normal BMI, Low Stress & 6.87\% \\
\textit{Baseline} & No Evidence & 8.73\% \\
\textit{Fragile} & Underweight, High Stress & 11.22\% \\
\textit{Overloaded} & Obese, High Stress & \textbf{30.18\%} \\
\textit{Perfect Storm} & Obese, Poor Sleep, High Load, Recent Inj. & \textbf{30.48\%} \\
\bottomrule
\end{tabular}
\end{adjustbox}
\caption{Predictive Risk Stratification. The model separates safe vs. critical states by a factor of $\approx 30x$.}
\label{tab:scenarios}
\end{table}

A key insight is the \textbf{Mechanical vs. Metabolic} comparison. While being underweight increases risk moderately (+28\% vs baseline), being obese under the same high-stress conditions causes risk to explode (+245\% vs baseline).

\textbf{Diagnostic Analysis.} By querying the causes of an injury event (Table \ref{tab:diagnostic}), we established a hierarchy of risk factors.

\begin{table}[h]
\centering
\small
\begin{adjustbox}{max width=\columnwidth}
\begin{tabular}{l|c|l}
\toprule
\textbf{Factor} & \textbf{P(Factor|Injury)} & \textbf{Interpretation} \\
\midrule
\textbf{Recent History} & \textbf{89.5\%} & Primary Cause (Recurrence) \\
\textbf{Low Readiness} & 69.4\% & Strong Early Warning \\
\textbf{Low HRV} & 43.1\% & Physiological Stress \\
\textbf{Neg. Lifestyle} & 37.5\% & Contextual Contributor \\
\textbf{High Load} & 21.2\% & Trigger (not sole cause) \\
\bottomrule
\end{tabular}
\end{adjustbox}
\caption{Posterior probabilities of risk factors given an injury.}
\label{tab:diagnostic}
\end{table}

The results reveal the \textbf{"Load Paradox"}. Only 21\% of injuries were associated with "High" acute workload spikes. The vast majority (89.5\%) were linked to a recent injury history, implying that most injuries are recurrences triggered by normal loads on a fragile system.

\section{Conclusion}
This work demonstrates that a Bayesian Network can effectively model athletic injury risk.
1. \textbf{Valid:} The model generalizes well to unseen data (Test Set Recall 69\%).
2. \textbf{Discriminatory:} It distinguishes between "Bulletproof" (1.1\%) and "Critical" (30.5\%) states.
3. \textbf{Actionable:} It identifies that prevention should prioritise managing return-to-play (History) over simple load restriction.

\section{Links to external resources}
\begin{itemize}
    \item Dataset: \url{https://zenodo.org/records/15401061}
    \item Reference Paper: Rossi \& Rodrigues (2025)
\end{itemize}

\bigskip
\bibliographystyle{aaai}
\bibliography{faikrmod3.bib}

\end{document}